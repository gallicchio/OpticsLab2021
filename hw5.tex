\documentclass[11pt]{hmcpset}
\usepackage[margin=1in]{geometry}
\usepackage{amsmath,amssymb,enumerate,graphicx,url}

\newcommand{\ket}[1]{|#1\rangle}
\newcommand{\bra}[1]{\langle #1|}
\newcommand{\braket}[2]{\langle #1|#2\rangle}
\newcommand{\braketop}[3]{\langle #1| #2 | #3 \rangle}
\newcommand{\normsq}[1]{\left|#1\right|^2}
\newcommand{\prob}[2]{\normsq{\braket{#1}{#2}}}
\newcommand{\avg}[1]{\langle #1 \rangle}


\usepackage{xcolor}
\usepackage{listings}

\definecolor{mGreen}{rgb}{0,0.6,0}
\definecolor{mGray}{rgb}{0.5,0.5,0.5}
\definecolor{mPurple}{rgb}{0.58,0,0.82}
\definecolor{backgroundColour}{rgb}{0.95,0.95,0.92}

\lstdefinestyle{Python}
{
	language=Python,
	backgroundcolor=\color{backgroundColour},   
	commentstyle=\color{mGreen},
	keywordstyle=\color{magenta},
	numberstyle=\tiny\color{mGray},
	stringstyle=\color{mPurple},
	basicstyle=\footnotesize\ttfamily,
	breakatwhitespace=false,         
	breaklines=true,                 
	captionpos=b,                    
	keepspaces=true,                 
	%numbers=left,                    
	numbersep=5pt,                  
	showspaces=false,                
	showstringspaces=false,
	showtabs=false,                  
	tabsize=3
}



\name{}
\class{PHYS134}
\assignment{Homework/Project 5}
\duedate{2021-03-22}

\begin{document}
\problemlist{}





%===========================================================
\begin{problem}[Damped Fabry–Perot Cavity (10\,pts)]
	Take the back-and-forth sum from lecture 16 and combine it with the phase shift accumulated by crossing the cavity ($kd$) to compute the intensity of light that exits the cavity in both directions along with the intensity of light inside of the cavity. Write your final result as a fraction of the intensity that is entering the cavity $I/I_0=?$. \\
	
	Check that energy is conserved: the incoming intensity must equal the outgoing intensities, both forward and backward. \\
	
	Make plots similar to those in lecture 15 and on page 117: as a function of frequency $\nu$, plot the fraction of the intensity of light that exits the cavity for different reflectivities: $|r|^2=50\%$, $|r|^2=90\%$, and $|r|^2=99\%$. (When we say a mirror is 50\% reflective, we mean that half of the \textit{intensity} reflects, because that's what we see and measure, not half of the wave amplitude.)	\\
	
	You can make the plot above assuming $r$ is real. Make additional plots for $|r|^2=90\%$, but allow $r$ to be complex to capture the case where there is a phase shift upon reflection. Pick a few phases to plot. Describe qualitatively what changes.	\\
	
	For each of these, calculate the ratio of intensity inside of the cavity to intensity exiting the cavity in the forward direction.
\end{problem}

\begin{solution}
	\vfill
\end{solution}
\pagebreak





\end{document}
