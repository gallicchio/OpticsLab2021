\documentclass[11pt]{hmcpset}
\usepackage[margin=1in]{geometry}
\usepackage{amsmath,amssymb,enumerate,graphicx,url}

\newcommand{\ket}[1]{|#1\rangle}
\newcommand{\bra}[1]{\langle #1|}
\newcommand{\braket}[2]{\langle #1|#2\rangle}
\newcommand{\braketop}[3]{\langle #1| #2 | #3 \rangle}
\newcommand{\normsq}[1]{\left|#1\right|^2}
\newcommand{\prob}[2]{\normsq{\braket{#1}{#2}}}
\newcommand{\avg}[1]{\langle #1 \rangle}


\usepackage{xcolor}
\usepackage{listings}

\definecolor{mGreen}{rgb}{0,0.6,0}
\definecolor{mGray}{rgb}{0.5,0.5,0.5}
\definecolor{mPurple}{rgb}{0.58,0,0.82}
\definecolor{backgroundColour}{rgb}{0.95,0.95,0.92}

\lstdefinestyle{Python}
{
	language=Python,
	backgroundcolor=\color{backgroundColour},   
	commentstyle=\color{mGreen},
	keywordstyle=\color{magenta},
	numberstyle=\tiny\color{mGray},
	stringstyle=\color{mPurple},
	basicstyle=\footnotesize\ttfamily,
	breakatwhitespace=false,         
	breaklines=true,                 
	captionpos=b,                    
	keepspaces=true,                 
	%numbers=left,                    
	numbersep=5pt,                  
	showspaces=false,                
	showstringspaces=false,
	showtabs=false,                  
	tabsize=3
}



\name{}
\class{PHYS134}
\assignment{Homework/Project 6}
\duedate{2021-03-29}

\begin{document}
\problemlist{}


%===========================================================
%\begin{problem}[Steck 8.1 (3\,pts)]
%	Write down the Jones matrix for an ideal polarizer that transmits $x$-polarized light. A realistic polarizer
%	along the $x$-direction blocks some fraction $\alpha$ (of the \textit{electric field}) of the $x$ polarization, and transmits some fraction $\beta$ of the $y$ polarization. (Both $\alpha$ and $\beta$ should be small for a decent polarizer.) Write down the Jones matrix for this realistic polarizer.
%\end{problem}
%
%\begin{solution}
%	\vfill
%\end{solution}
%\pagebreak


%===========================================================
\begin{problem}[Steck 8.2 (5\,pts)]
	\begin{enumerate}
		\item[(a)] Consider a linear polarizer and a wave linearly polarized at an angle $\theta$ with respect to the polarizer's transmission axis. Show that the intensity of the wave is reduced by $\cos^2 \theta$ after passing through the polarizer (this is called the \textbf{Law of Malus}). [Hint: Use the rotation matrix and its inverse.]
		\item[(b)] Consider a system of $N$ cascaded polarizers. The polarizers have their transmission axes at angles $\pi/2N, 2\pi/2N, 3\pi/2N, \cdots \pi/2$ from the $x$-axis, in the order that an input wave sees them. That is, the last polarizer is oriented along the $y$-direction. Suppose that an input wave is polarized in the $x$-direction. Compute the intensity transmission coefficient for the system. Show that the transmission coefficient approaches unity as $N \rightarrow \infty$. This is a simple realization of the \textbf{quantum Zeno effect}, where each polarizer acts as a ``measurement'' of the polarization state—the polarization is ``dragged'' by the measurements as long as they are sufficiently frequent.
	\end{enumerate}
\end{problem}

\begin{solution}
	\vfill
\end{solution}
\pagebreak

%===========================================================
\begin{problem}[Arbitrary Waveplate Jones Matrices (10\,pts)]
	\begin{enumerate}
		\item[(a)] Write the Jones Matrix for an arbitrary wave plate, which adds a phase $\eta$ in the y-direction relative to the x-direction.
		\item[(b)] Find the Jones Matrix for such a waveplate when it's oriented at some angle $\theta$. [For this and what follows, something like Mathematica is helpful. And don't worry about the sign of $\theta$ that I worried about in my lecture---everything here is even in $\theta$.]
		\item[(c)] If horizontally-polarized light passes through this waveplate and then thorugh a horizontal polarizer, what fraction of intensity makes it through as a function of $\theta$ and $\eta$?
		\item[(d)] Simplify this fraction as a function of $\theta$ for a half-waveplate and a quarter-waveplate.
		\item[(e)] Plot this fraction as a function of $\theta$ for three different cases on the same plot (with a  legend): half-waveplate, quarter-waveplate, and eighth-waveplate.
	\end{enumerate}
\end{problem}

\begin{solution}
	\vfill
\end{solution}
\pagebreak



%===========================================================
\begin{problem}[Fitting Waveplate Data (10\,pts)]
	Watch the video \texttt{Lab5Polarization.mp4}. I wasn't careful enough with the polarizer and HWP data, and the motorized data at the end had some strange features, so I'll only ask you to analyze the quarter-waveplate data. The others would have been boringly similar. \\
	
	Download \texttt{QWP.csv}. Briefly look at the file in a text editor or spreadsheet program. Fit the data to the function you found in the previous problem for the arbitrary waveplate. You'll need 3 parameters:
	\begin{itemize}
		\item the overall amplitude (the meter reads in mW, not fraction transmitted),
		\item the angular offset of the waveplate in the mount (a few degrees plus or minus---by convention, this parameter is meant to be the reading on the dial that sets the actual wave-plate to true zero degrees), and
		\item the fractional retardance (should be close to 0.25 because it would be a quarter waveplate if the wavelength of the laser diode exactly matched the wavelength that the waveplate was designed for, which is almost never perfectly the case).
	\end{itemize}
	Be sure to supply reasonable, non-zero initial guesses for these parameters because this is a non-linear fit. \\
	
	Be sure to calculate the reduced $\chi^2$ and PTE.
\end{problem}

\begin{solution}
	\vfill
\end{solution}
\pagebreak





\end{document}
